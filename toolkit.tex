\documentclass[11pt]{report}
%\usepackage[utf8]{inputenc}
\usepackage[margin=1in]{geometry}
\usepackage{amsfonts,amsthm,amsmath}
\usepackage{mathtools}
\usepackage{complexity}
\usepackage[chapter]{algorithm}
\usepackage{algpseudocode,caption} 
\usepackage{graphicx}
\usepackage{relsize}
\usepackage{tikz}  %TikZ central library is called.
%\usepackage{tkz-graph}
%	\usepackage{tkz-berge}
%\usepackage{tikz-network}
\usetikzlibrary{automata,positioning,calc}

\usepackage{tocloft}
\usepackage{palatino}

\RequirePackage[colorlinks=true]{hyperref}
\hypersetup{
  linkcolor=[rgb]{0.3,0.3,0.6},
  citecolor=[rgb]{0.2, 0.6, 0.2},
  urlcolor=[rgb]{0.6, 0.2, 0.2}
}

\usepackage{setspace}
\onehalfspacing


\usepackage{xcolor}
\usepackage{color}
\definecolor{delta}{rgb}{0,0.2,0}
\definecolor{gamma}{rgb}{0,0,0.2}
\definecolor{beta}{rgb}{0.2,0,0}
\definecolor{alpha}{rgb}{0.8,0,0}
\newcommand{\wt}[1]{\widetilde{#1}}
\newcommand{\sse}{\subseteq}
\newcommand{\zo}{\{0,1\}}
\newcommand{\zon}{\zo^n}
\newcommand{\aphantom}{\vphantom{2^2}}
\newcommand{\aaphantom}{\vphantom{2^{2^2}}}
\newcommand{\defn}{\stackrel{\text{\tiny def}}{=}}

% left-right wrappers
\newcommand{\set}[1]{\left\{ #1 \right\}}
\newcommand{\card}[1]{\left|#1 \right|}
\newcommand{\mytilde}[1]{\overset{\sim}{#1}}

% latin
\newcommand{\etal}{\textit{et al}.\@\xspace}
\newcommand{\ie}{i.e.}
\usepackage[all]{xy}
\usepackage{setspace}
\usepackage{amssymb}
\usepackage{amsmath}
\newtheoremstyle{myplain}{10pt}{10pt}{\itshape}{}{\scshape}{.}{.5em}{}
\newtheoremstyle{mydefinition} {10pt}{10pt}{\itshape}{}{\scshape}{.}{.5em}{}
\newtheoremstyle{myremark}     {10pt}{10pt}{}{}{\scshape}{.}{.5em}{}
\newcounter{lecture}
\theoremstyle{myplain}

\newtheorem{theorem}{Theorem}[section]
\newtheorem{lemma}[theorem]{Lemma}
\newtheorem{proposition} [theorem]{Proposition}
\newtheorem{corollary}[theorem]{Corollary}
\newtheorem{claim}[theorem]{Claim}
\newtheorem{fact} [theorem]{Fact}
\theoremstyle{mydefinition}
\newtheorem{definition} [theorem]{Definition}
\newtheorem{example}[theorem]{Example}
\newtheorem{assumption}[theorem]{Assumption}
\newtheorem{openproblem}[theorem]{Open Problem}
\newtheorem{problem} [theorem]{Problem}
\theoremstyle{myremark}
\newtheorem{remark} [theorem]{Remark}
\newtheorem{conjecture} [theorem]{Conjecture}
\newtheorem{observation}[theorem]{Observation}
%\newtheorem{exercise}{Exercise}

\numberwithin{equation} {lecture}
\numberwithin{figure}{lecture}
\numberwithin{table}{lecture}
\newcommand{\cupdot}{\mathbin{\mathaccent\cdot\cup}}
\newcommand{\bigsum}{\mathlarger{\mathlarger{\sum}}}
\newcommand{\bigger}[1]{\mathlarger{\mathlarger{#1}}}
\renewcommand{\bar}[1]{\overline{\vphantom{1^a}#1}}
%\renewcommand{\fnum@figure}{\textsc{Figure~\thefigure}}
%\renewcommand{\fnum@table}{\textsc{Table~\thetable}}

\theoremstyle{plain}

\newenvironment{proof-sketch}{\noindent{\bf Sketch of Proof}\hspace*{1em}}{\qed\bigskip}
\newenvironment{proof-idea}{\noindent{\bf Proof Idea}\hspace*{1em}}{\qed\bigskip}
\newenvironment{proof-of-lemma}[1]{\noindent{\bf Proof of Lemma #1}\hspace*{1em}}{\qed\bigskip}
\newenvironment{proof-attempt}{\noindent{\bf Proof Attempt}\hspace*{1em}}{\qed\bigskip}
\newenvironment{proofof}[1]{\noindent{\bf Proof}
of #1:\hspace*{1em}}{\qed\bigskip}

\renewcommand{\qedsymbol}{\leavevmode
  \hbox to.77778em{%
  \hfil\vrule
  \vbox to.875em{\hrule width.35em\vfil\hrule}%
  \vrule\hfil}}

%%%%%%%%%%%%%%%%%%%%%%%%%%%%%%%%%%%%%%%%%%%%%%%%%%%
% Useful Macros
%%%%%%%%%%%%%%%%%%%%%%%%%%%%%%%%%%%%%%%%%%%%%%%%%%%
\newcommand{\Exp}[1]{\mathify{\mbox{Exp}\left[#1\right]}}
\newcommand{\bigO}O
%\newcommand{\set}[1]{\mathify{\left\{ #1 \right\}}}
\def\half{\frac{1}{2}}
\newcommand{\V}[1]{\mathsf{Var}[#1]}
\def\implies{\Rightarrow}
\def\prob#1#2{{\mathop{{\rm Prob}}_{#1}}\left[#2 \right]}
\def\var#1#2{{\mathop{{\rm Var}}_{#1}}[#2]}
\def\expec#1#2{{\mathop{{\rm E}}_{#1}}[#2]}
\def\sizeof#1{\left| #1\right|}
\def\setof#1{\left\{ #1\right\}  }
\newcommand\norm[1]{{\left\lVert#1\right\rVert}_2}
\newcommand{\F}{{\mathbb{F}}}
\newcommand{\Z}{{\mathbb{Z}}}
\newcommand{\supp}{{\mathsf{supp}}}
%\newcommand{\qed}{\rule{7pt}{7pt}}
\newcommand*\circled[1]{~\tikz[baseline=(char.base)]{
            \node[shape=circle,draw,inner sep=1.5pt] (char) {\tiny #1};}~}
           
\newcommand{\FOR}{{\bf for}}
\newcommand{\TO}{{\bf to}}
\newcommand{\DO}{{\bf do}}
\newcommand{\WHILE}{{\bf while}}
\newcommand{\AND}{{\bf and}}
\newcommand{\IF}{{\bf if}}
\newcommand{\THEN}{{\bf then}}
\newcommand{\ELSE}{{\bf else}}
\newcommand{\N}{\mathbb{N}}

% \renewcommand{\thefigure}{\thesection.\arabic{figure}}
% \renewcommand{\thetable}{\thesection.\arabic{table}}
% \renewcommand{\theequation}{\thesection.\arabic{equation}}

% Calligraphic letters
\newcommand{\calA}{{\cal A}}
\newcommand{\calB}{{\cal B}}
\newcommand{\calC}{{\cal C}}
\newcommand{\calD}{{\cal D}}
\newcommand{\calE}{{\cal E}}
\newcommand{\calF}{{\cal F}}
\newcommand{\calG}{{\cal G}}
\newcommand{\calH}{{\cal H}}
\newcommand{\calI}{{\cal I}}
\newcommand{\calJ}{{\cal J}}
\newcommand{\calK}{{\cal K}}
\newcommand{\calL}{{\cal L}}
\newcommand{\calM}{{\cal M}}
\newcommand{\calN}{{\cal N}}
\newcommand{\calO}{{\cal O}}
\newcommand{\calP}{{\cal P}}
\newcommand{\calQ}{{\cal Q}}
\newcommand{\calR}{{\cal R}}
\newcommand{\calS}{{\cal S}}
\newcommand{\calT}{{\cal T}}
\newcommand{\calU}{{\cal U}}
\newcommand{\calV}{{\cal V}}
\newcommand{\calW}{{\cal W}}
\newcommand{\calX}{{\cal X}}
\newcommand{\calY}{{\cal Y}}
\newcommand{\calZ}{{\cal Z}}


\setcounter{tocdepth}{3}
\setcounter{secnumdepth}{2}
\sloppy

\newcommand{\AsymCloud}[3]{
\begin{scope}[shift={#1},scale=#3]
\draw (-1.6,-0.7) .. controls (-2.3,-1.1)
and (-2.7,0.3) .. (-1.7,0.3)coordinate(asy1) .. controls (-1.6,0.7)
and (-1.2,0.9) .. (-0.8,0.7) .. controls (-0.5,1.5)
and (0.6,1.3) .. (0.7,0.5) .. controls (1.5,0.4)
and (1.2,-1) .. (0.4,-0.6)coordinate(asy2) .. controls (0.2,-1)
and (-0.2,-1) .. (-0.5,-0.7) .. controls (-0.9,-1)
and (-1.3,-1) .. cycle;
\node at ($(asy1)!0.5!(asy2)$) {#2};
\end{scope}
}



\AtBeginDocument{\renewcommand\contentsname{Table of Contents}}

\newcommand{\listofscribes}{List of Scribes}
\newcommand{\listofinstr}{List of Instructors}
\newlistof{scribe}{scr}{\listofscribes}
\newlistof{instructors}{instr}{\listofinstr}

\newcommand{\Lecture}[7]{
\newpage
\setcounter{chapter}{#3}
\setcounter{lecture}{#3}

% To reset numbering in a new lecture
\setcounter{theorem}{0}  
% To reset numbering within a section
\setcounter{section}{0}  

% For table of contents
\cleardoublepage	% for two sided document with odd no. of TOC pages
\phantomsection	% for fixing hyperref link
\addcontentsline{toc}{chapter}{Lecture \protect\numberline{#3}(#6) {#4}}

% For list of scribes
\cleardoublepage	% for two sided document with odd no. of TOC pages
\phantomsection		% for fixing hyperref link

% Adding the correctly colored line to scribe list.

\ifthenelse{\equal{#6}{$\alpha$}}{\addcontentsline{scr}{scribe}{\protect {\sf Lecture } \numberline{\sf \thelecture} {\color{alpha}{#5} - (#6) \tiny{TA : #7}}}}
{
\ifthenelse{\equal{#6}{$\beta$}}{\addcontentsline{scr}{scribe}{\protect {\sf Lecture } \numberline{\sf \thelecture} {\color{beta}{#5} - (#6) \tiny{TA : #7}}}}
{
\ifthenelse{\equal{#6}{$\gamma$}}{\addcontentsline{scr}{scribe}{\protect {\sf Lecture } \numberline{\sf \thelecture} {\color{gamma}{#5} - (#6) \tiny{TA : #7}}}}
{
\ifthenelse{\equal{#6}{$\delta$}}{\addcontentsline{scr}{scribe}{\protect {\sf Lecture } \numberline{\sf \thelecture} {\color{delta}{#5} - (#6) \tiny{TA : #7}}}}
{
\addcontentsline{scr}{scribe}{\protect {\sf Lecture } \numberline{\sf \thelecture} {{#5} - (Incorrect Labelling) \tiny{TA: #7}}}
}}}}

%\addcontentsline{scr}{scribe}{\protect {\sf Lecture } \numberline{\sf \thelecture} {\em #5} (#6)}

% For list of instructors
\cleardoublepage	% for two sided document with odd no. of TOC pages
\phantomsection		% for fixing hyperref link
\addcontentsline{instr}{instructors}{\protect {\sf Lecture } \numberline{\sf \thelecture} {\em #1}}
\noindent
\parbox{9cm}{
%   Indian Institute of Technology Madras\\
%	CS5130 - Mathematical Tools for Theoretical Computer Science\\
%	\rule{0mm}{6mm}%
	{\it \bf Instructor :} #1 \\
	{\it \bf Scribe :} #5 ({\it TA:} #7) \\
	{\it \bf Date :}     #2 \\
	{\it \bf Status :} #6
}
\hfill
\begin{tabular}{c@{}}
{\bf\Large Lecture}\\
\rule{0mm}{17mm}\scalebox{6.6}{\bf\thelecture}
\end{tabular}

\vspace{1.5cm}
\begin{center}
{\Large \bf #4}
\end{center}
\vspace{1cm}
%\thispagestyle{empty}
}

%
%\newpage
%\listofscribe          % For automatic scribe list generation.
%
%\newpage
%\listofinstr
%\newpage
%\tableofcontents
%
%\newpage 
%\pagenumbering{arabic}  % Arabic page numbering for lectures.
%%\part{Introduction, Motivation and the Language}
%\newpage \setcounter{page}{1} 

\usepackage{collect}
\usepackage{hypcap}
\renewcommand{\E}{{\mathbb E}}

\usepackage{todonotes}
\newcounter{todocounter}
\newcommand{\todonum}[2][]{\stepcounter{todocounter}\todo[#1]{\thetodocounter: #2}}
\newcommand{\jsay}[1]{\todonum[inline,color=red!20]{\small Jayalal says: Todo - #1}}

% modified exercise enviornment to use with collect  package
\usepackage{enumerate}
\newcounter{excount}
\setcounter{excount}{0}
\theoremstyle{definition}
\newtheorem{ex}[section]{Exercise}
\newtheorem{curious}[theorem]{Curiosity}
% for exercises which are problem set questions

\newtheorem{exercise-prob}[section]{Exercise}
\theoremstyle{plain}	    
%%%%%%%%%%%%%%%%%%%% Exercise and pset macros (start) %%%%%%%%%%%%%%%%%%%%%%%5
% For problem set back reference.
\def\psetbackref{1}

% For Curiosity Drive
\definecollection{curious.tmp}
\makeatletter
\newenvironment{curiousity}
    {\@nameuse{collect*}{curious.tmp}
		  {\begin{curious}}
		    {\end{curious}}{}{}
    }{\@nameuse{endcollect*}}
\makeatother

% For exercise
\definecollection{ex.tmp}
\makeatletter
\newenvironment{exercise}
    {\@nameuse{collect*}{ex.tmp}
		  {\begin{ex}}
		    {\end{ex}}{}{}
    }{\@nameuse{endcollect*}}
\makeatother


%%%%%%%%%%%%%%%%%%%%%%%%%%%%%%%%%%%%%%%%%%%%%%%%%%%%%%%%%%%%%%
% To create a a new pset with pset number n, copy paste the following code
% with XX replaced by n.
%
% 	 \definecollection{psXX.tmp}
%	 \makeatletter
%	 \newenvironment{show-psXX}[1]
%	     {\@nameuse{collect*}{psXX.tmp}
%	         {\ifthenelse{ \equal{\psetbackref}{1} }{\label{prob:#1}}{}} {}
%   	         {\item \label{#1} (See Exercise~\ref{prob:#1})} {}
%	     }{\@nameuse{endcollect*}}
%	 \makeatother
%
%	 \makeatletter
%	 \newenvironment{psXX}
%	     {\@nameuse{collect}{psXX.tmp}
%			{\item}{}
%	     }{\@nameuse{endcollect}}
%	 \makeatother
%
% 

%%% For ps1 
\definecollection{ps1.tmp}
\makeatletter
\newenvironment{show-ps1}[1]
    {\@nameuse{collect*}{ps1.tmp}
	    {\ifthenelse{ \equal{\psetbackref}{1} }			{\label{prob:#1}}{}} {}
	    {\item \label{#1} (See Exercise~\ref{prob:#1})} {}
    }{\@nameuse{endcollect*}}
\makeatother

\makeatletter
\newenvironment{ps1}
    {\@nameuse{collect}{ps1.tmp}
		{\item }{}
    }{\@nameuse{endcollect}}
\makeatother

%%% For ps2
\definecollection{ps2.tmp}
\makeatletter
\newenvironment{show-ps2}[1]
    {\@nameuse{collect*}{ps2.tmp}
	    {\ifthenelse{ \equal{\psetbackref}{1} }{\label{prob:#1}}{}} {}
	    {\item \label{#1} (See Exercise~\ref{prob:#1})} {}
    }{\@nameuse{endcollect*}}
\makeatother

\makeatletter
\newenvironment{ps2}
    {\@nameuse{collect}{ps2.tmp}
		{\item}{}
    }{\@nameuse{endcollect}}
\makeatother

%%% For ps3
\definecollection{ps3.tmp}
\makeatletter
\newenvironment{show-ps3}[1]
    {\@nameuse{collect*}{ps3.tmp}
	    {\ifthenelse{ \equal{\psetbackref}{1} }{\label{prob:#1}}{}} {}
	    {\item \label{#1} (See Exercise~\ref{prob:#1})} {}
    }{\@nameuse{endcollect*}}
\makeatother

\makeatletter
\newenvironment{ps3}
    {\@nameuse{collect}{ps3.tmp}
		{\item}{}
    }{\@nameuse{endcollect}}
\makeatother


%%% For ps3
\definecollection{ps4.tmp}
\makeatletter
\newenvironment{show-ps4}[1]
    {\@nameuse{collect*}{ps4.tmp}
	    {\ifthenelse{ \equal{\psetbackref}{1} }{\label{prob:#1}}{}} {}
	    {\item \label{#1} (See Exercise~\ref{prob:#1})} {}
    }{\@nameuse{endcollect*}}
\makeatother

\makeatletter
\newenvironment{ps4}
    {\@nameuse{collect}{ps4.tmp}
		{\item}{}
    }{\@nameuse{endcollect}}
\makeatother

%%%%% Exercise and pset macros (end) %%%%%

\renewcommand{\E}{{\mathbb E}}
\title{{\huge CS5130 : Mathematical Tools for Theoretical Computer Science} \\[3mm]
{\LARGE(Scribe Lecture Notes)}\\[1cm]}
\author{{\Large Lecturer : {\sc Jayalal Sarma}} \\[3mm]
Department of Computer Science and Engineering \\[1mm]
Indian Institute of Technology Madras (IITM)\\[1mm]
Chennai, India}
\date{Last updated on : \today}


\begin{document}
\maketitle
\setcounter{page}{1}

\newpage
\pagenumbering{roman}  % Roman numbering in intro portion.
\chapter*{Preface}

This lecture notes are produced as a part of the course \textsf{CS5130: Mathematical Tools for Theoretical Computer Science} which was a course offered (during the online semester Sep-Dec 2020) at the CSE Department of IIT Madras.

\section*{Acknowledgements}

We acknowledge the efforts of the scribes and editors of this document.

%\listofscribe

\section*{Scribe status}
Each lecture has a field called {\bf status}. It tells which stage of the edit
pipeline is the document currently. The scribe notes are due on Thursdays and Saturdays as per the following timeline.

\begin{center}
\includegraphics[scale=1.5]{scribe-timeline.pdf}
\end{center}

Even after these edits, it is possible that there are still errors in the draft, which may not get noticed. If you find errors still, please report to the instructor.


\newpage
\listofscribe
\newpage
\tableofcontents
\newpage
\listoftodos
\setcounter{page}{1}
\setstretch{1.1}

\input{lecture01.tex}
\Lecture{Jayalal Sarma}{Sep 15, 2020}{02}{Pigeon Hole Principle - More Applications}{Jayalal Sarma}{$\alpha$}{Jayalal Sarma}

\section{Example 1}
\section{Example 2}
\section{Example 3}
%\Lecture{3}{Yet to be added}

%\Lecture{4}{Yet to be added}

%\Lecture{5}{Yet to be added}

%\Lecture{Jayalal Sharma}{Sept 19, 2020}{06}{Catlan Bijections}{Anshu and Narasimha Sai}{$\alpha$}{JS}

\section{Introduction}
One of the classic examples to demonstrate the power of bijections is \emph{Catlan numbers}. The Catlan numbers form a sequence of natural numbers that occur in various counting problems and occurs in several seemingly different contexts. Historically, \emph{Euler} is the first person to study them. He was interested in counting the number of ways of dividing a polygon into triangles by drawing non-overlapping diagonals. Catlan numbers got their name from \emph{Eugene Catlan} when he used them to answer the \emph{Parenthesisation problem} which is the following: Consider a sequence $(a_1,a_2,\cdots,a_{n+1})$ of $n+1$ numbers, If we have to perform a binary operations $\odot$ $n$ times among them, how many number of ways are there to parenthesise (or bracket) them using $n$ parenthesis of single type (say $'()'$). In this lecture, we will see a few equivalent problems to this and then arrive at an explicit expression of Catlan numbers.
\section{Equivalent Bijections}
In this section, we see a few equivalent problems of the \emph{parenthesisation} problem and argue that answer to each of them is also the \emph{catlan number}
\paragraph{Full binary trees} If we observe the Parenthesisation problem carefully, we notice that every valid parentesisation of those $n+1$ numbers form a \emph{full binary tree} (a binary tree in which every node have either two children or no children) of $n+1$ leaves and $n$ internal nodes where leaves represents the numbers $a_1,\cdots,a_{n+1}$ and each internal node corresponds to one operation. Therefore, there's an implicit bijection between the set of valid parenthesisations and full binary trees with $n$ internal nodes. Therefore, 
\begin{equation}
    \substack{\textrm{number of valid parenthesisations of }\\ n+1 \textrm{ elements }}  = \substack{\textrm{number of full binary trees with }\\ n \textrm{ internal nodes}}
\end{equation}  

\paragraph{Balanced parenthesised strings} A balanced parenthesised string of length $2n$ is a string consists of $n$ left brackets $'('$ and $n$ right brackets $')'$ in which every prefix of the string has number of left brackets $'('$ $\geq$ number of right brackets $')'$. One can easily observe the bijection from set of balanced paranthesised string to valid parenthesisations of $n+1$ numbers

\paragraph{Euler's problem} Find the number of ways of triangulating a polygon with $n+2$ edges

\paragraph{Handshaking problem} Consider a scenario where $2n$ people are sitting around a table. How many ways they can shake hands with each other without crossing hands. We leave it as an exercise to establish bijections from \emph{Euler's} problem to \emph{Full binary tree} problem and \emph{handshaking} problem to \emph{balanced parenthesised strings} problem.
\jsay{Establish bijections from \emph{Euler's} problem to \emph{Full binary tree} problem and \emph{handshaking} problem to \emph{balanced parenthesised strings} problem}

\section{Algebraic Expression}
In this section, we are interested in arriving at a concrete expression of the $n^{th}$ \emph{catlan number} (denoted by $c_n$). Let's solve another problem and then, by establishing a bijection to one of the above problems, we can arrive at an expression for $c_n$.

\subsection{Monotone walk on $n\times n$ grid} Suppose we have a grid of size $n\times n$. How many ways are there to go from $(0,0)$ to $(n,n)$ by using only downward edges or right edges. A sample path is represented in Fig. \ref{fig:sample-path}. We observe that each step can increment the value of exactly one of the co-ordinates by $1$. Since we have to move from $(0,0)$ to $(n,n)$, we have to increase the value of both the co-ordinates by $n$ and $n$ and thus irrespective of the path you take, the length of a path from $(0,0)$ to $(n,n)$ must be of length $n+n=2n$.

\begin{figure}[h!]
    \centering
    \includegraphics[width=0.4\linewidth]{sample-path.png}
    \caption{A path from $(0,0)$ to $(n,n)$ using downward and right edges}
    \label{fig:sample-path}
\end{figure}

If we represent each right move as $R$ and each downward move as $D$, one can observe that there's a bijection $f$ from the set of paths to set of strings of length $2n$ over the alphabet $\{D,R\}$ with number of $D$'s = number of $R$'s = $n$. Formally, if $(u_0,v_0), (u_1,v_1),\cdots,(u_{2n},v_{2n})$ represents the path where $(u_0,v_0)=(0,0)$ and $(u_{2n},v_{2n})=(n,n)$, and $b=b_1b_2\cdots b_{2n}$ represents the string where each $b_i$ is either $D$ or $R$, our bijection $f$ takes a path as input and sets $b_i$ as
$$b_i=\begin{cases}
D &\mbox{if } u_i = u_{i-1}+1\\
R &\mbox{if } v_i = v_{i-1}+1
\end{cases}$$
\begin{description}
\item \underline{Well defined:} As we have exactly $n$ $x$ co-ordinate increments and $n$ $y$ co-ordinate increments, we will have exactly $n$ $D$'s and $n$ $R$'s in our string and thus $f$ is well defined.
\item \underline{Injective:} Two different paths from $(0,0)$ to $(n,n)$ will different in at least one $(u_{i-1},v_{i-1})$ to $(u_i,v_i)$ transition where $i=1,2,\cdots,2n$, their corresponding strings under $f$ will differ in at least $i^{th}$ position and thus $f$ is injective.
\item \underline{Surjective:} Every string over $\{D,R\}$ of length $2n$ with equal number of $D$'s and $R$'s has a pre-image under $f$ which is defined by $(u_0,v_0)=(0,0)$ and $(u_i,v_i)$ is $(u_{i-1}+1,v_{i-1})$ if $b_i=R$ and $(u_{i-1},v_{i-1}+1)$ if $b_i=D$. As there will be $n$ $D$'s and $n$ $R$'s, $(u_{2n},v_{2n})=(n,n)$ and thus $f$ is surjective .
\end{description} 
Thus $f$ is bijection. As we have number of string over $\{D,R\}$ of length $2n$ with equal number of $D$'s and $R$'s equal to $\binom{2n}{n}$ (select $n$ positions out of $2n$ available and fill them with $D$'s and the rest with $R$'s). Thus the number of paths from $(0,0)$ to $(n,n)$ with only downward and rightward movements is $\binom{2n}{n}$.

Lets ask a slightly question. How many ways are there to go from $(0,0)$ to $(n+1,n-1)$ using only downward or right edges.Using a similar arguments as above, we can come up with a bijection to set of string over $\{D,R\}$ of length $2n$ with $n+1$ $D$'s and $n-1$ $R$'s. Therefore number of required paths are $\binom{2n}{n+1}=\binom{2n}{n-1}$


%\Lecture{7}{Yet to be added}

%\Lecture{8}{Yet to be added}

%\Lecture{9}{Yet to be added}

%\Lecture{10}{Yet to be added}

%\Lecture{11}{Yet to be added}

%\Lecture{12}{Yet to be added}

%\Lecture{13}{Yet to be added}

%\Lecture{14}{Yet to be added}

%\Lecture{15}{Yet to be added}

%\Lecture{16}{Yet to be added}

%\Lecture{17}{Yet to be added}

%\Lecture{18}{Yet to be added}

%\Lecture{19}{Yet to be added}

%\Lecture{20}{Yet to be added}

%\Lecture{21}{Yet to be added}

%\Lecture{22}{Yet to be added}

%\Lecture{23}{Yet to be added}

%\Lecture{24}{Yet to be added}

%\Lecture{25}{Yet to be added}

%\Lecture{26}{Yet to be added}

%\Lecture{27}{Yet to be added}

%\Lecture{28}{Yet to be added}

%\Lecture{29}{Yet to be added}

%\Lecture{30}{Yet to be added}

%\Lecture{31}{Yet to be added}

%\Lecture{32}{Yet to be added}

%\Lecture{33}{Yet to be added}

%\Lecture{34}{Yet to be added}

%\Lecture{35}{Yet to be added}

%\Lecture{36}{Yet to be added}

%\Lecture{37}{Yet to be added}

%\Lecture{38}{Yet to be added}

%\Lecture{38}{Yet to be added}

%\Lecture{40}{Yet to be added}

%\Lecture{41}{Yet to be added}

%\Lecture{42}{Yet to be added}


%\input{problemsets.tex}

%\part{Exercise \& Problem Sets}
%\chapter{Exercises}
%\newpage
%\section{Exercises}
%\setcounter{excount}{0}
%\includecollection{ex.tmp}

\newpage

%\section{~Curiosity Drive}
%Here we list down all the "out of curious" questions that we discussed (sometimes even not discussed) in the class (and hence in this document).
%\includecollection{curious.tmp}
%
%% Convention : Call all problem set collection names with psXX.tmp This helps in managing the auxillary files created by collect package.Give back reference to problems in lectures.
%\def\psetbackref{0}
%
%\chapter{Problem Sets}

%\section{~Problem Set \#1}
%\begin{enumerate}[(1)]
%\includecollection{ps1.tmp}
%\end{enumerate}
%
%\newpage
%\section{~Problem Set \#2}
%
%\begin{enumerate}[(1)]
%\includecollection{ps2.tmp}
%\end{enumerate}
%
%\newpage
%\section{~Problem Set \#3}
%
%\begin{enumerate}[(1)]
%\includecollection{ps3.tmp}
%\end{enumerate}
%
%\newpage
%\section{~Problem Set \#4}
%
%\begin{enumerate}[(1)]
%\includecollection{ps4.tmp}
%\end{enumerate}

\bibliographystyle{apalike}
\bibliography{references}

\end{document}
